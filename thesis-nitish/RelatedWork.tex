\chapter{Related Work}
\label{chapter:relatedwork}
The task of text classification, i.e. classification of documents into a fixed number of predefined categories has been long studied in-depth for many years now. This multi-class classification problem has further evolved into a multi-label text classification task where each document can belong to multiple, exactly one or no category at all. 

Supervised machine learning techniques that learn classifiers to perform this category assignment task can be broken down into two main components, namely, text representation and learning algorithm. 
Text representation involves converting the documents, that are usually strings of characters, into numerical vectors that are suitable inputs to the learning algorithm while the learning algorithm uses pairs of labeled input text representations and the categories it is belongs in, to learn a model so as to classify new documents into categories.

\section{Text Representation}
\label{sec:textrepr}
Any text-based classification system requires the documents to be represented in an appropriate manner dictacted by the task being performed \citep{lewis1992text}. Moreover, \citep{quinlan1983learning} showed that the accuracy of the classification task depends as much on the document representation as on the learning algorithm being employed. Different from the data mining task, which deals with structureed documents, text classification deals with unstructured documents that need to be appropriately transformed into numerical vectors, i.e. the need for text representation. In this section we introduce the most effective and widely-used techniques to represent documents for text classification.

\subsection{Bag of Words}
It is found in information retrieval research that word stems work well as representations units for documents and that their ordering in a document is of minor importance for many tasks. This is attributed by the fact that the most widely-used used model to represent documents for the classification task is the \emph{Vector Space Model (VSM)} \citep{salton1973specification}. 

In the Vector Space Model, a document $d$ is represented as a vector in the term/word space, $d$ $=$ $(w_{1}, w_{2}, \ldots, w_{|V|})$ where $|V|$ is the size of the vocabulary. Each of the $w_{i} \in \left[0,1\right]$, represents the weightage of the term $i$ in the document $d$. This is called the \emph{bag-of-words} model as it ignores word ordering and each document is reduced to a bag of words that it contains or not. 

An important requirement of such a representation is that, the terms that help in defining the semantic content of the document and play an important role in classification be given higher weightage than the others. Over the years, there has been much research in the information retrieval field on term weighting schemes. The most important term-weighting techniques are described below : 
\begin{enumerate}
\item{\textbf{One Hot Representation} : }This is the most trivial representation, where each document is represented by a vector that is size of the vocabulary. Each element in the vector is either a $0$ or a $1$ to denote the absence or presence of a specific term in the document.

\item{\textbf{Term Frequency (tf))} : }The term frequency representation weighs the terms present in the document relative to their occurence frequency in the document. Hence a document $d$ is represented as, $d$ $=$ $(w_{1}, w_{2}, \ldots, w_{|V|})$, where, $w_{k}$ is the number of times the term $k$ appears in the document $d$. 
% \begin{equation}
% w_{i} = \frac{N_{i,d}}{N_{d}}
% \end{equation}
% where, $N_{i,d}$ is the number of times, term $i$ occurs in document $d$ and $N_{d}$ is the total number of terms in the document. The document can also be represented only by the counts of terms, but normalization is done to reduce the effect of document length.
\item{\textbf{Inverse Document Frequency (idf)} : }Though using \emph{tf} as a term weighting scheme is a good starting point, it faces a challenge when high frequency terms are not concentrated in a few particular documents but are prevalent in the whole collection. Those terms then stop being characteristic of the semantic content of a few documents and need not be given high weightage. To overcome this problem, \cite{salton1988term} suggested a new term weighting called the inverse document frequency (idf). The \emph{idf} weight of a term varies inversely with the number of documents $n$ it belongs to in a collection of total $N$ documents. A typical \emph{idf} vector can be computed as 
\begin{equation}
w_{k} = \log \frac{N}{n}
\end{equation}
\item{\textbf{Term Frequency Inverse Document Frequency (tf-idf)} : }Given the above two term weighing schemes, it is clear that an important term in a document should have high \emph{tf} but a low overall collection frequency (\emph{idf}). This suggests that a reasonable measure for term importance may be then obtained by the \emph{tf} and the \emph{idf} (\emph{tf}$\times$\emph{idf}). As we will see in the results section, the \emph{tf-idf} weighed bag-of-words document representation gives one of the best accuracies in the multi-label text classification task.
\end{enumerate}
A common feature in the bag-of-words document representation is the \emph{normalization factor}\citep{salton1988term} introduced to reduce the effect of varying document lengths and give equal weightage to documents of all lengths when learning the classifier for text categorization. \todo{Do we put how normalization is done?}
Another feature added to the bag-of-words representation is the removal of stop-words (short function words that do not add to the semantic content of the document) and words that occur infrequently to make the document vector more meaningful.

\subsection{Dimensionality Reduction / Feature Selection}
The bag-of-words representation scheme has several drawbacks but the most important drawback it suffers from is that document vectors are very sparse and high dimensional. Typical vocabulary sizes of a moderate-sized document collection ranges from tens to hundereds of thousands of terms which is prohibitively high for many learning algorithms. 
To overcome this issue of high-dimensional bag-of-words document representations, automatic feature selection is performed that removes uninformative terms according to corpus statistics and constructs new orthogonal features by combining several lower level features (terms/words). Several techniques used in practice are discussed below, 
\begin{enumerate}
\item{\textbf{Information Gain} : }Information Gain is widely used as a term-goodness criterion in the field of machine learning, mainly in decision trees \citep{quinlan1986induction} and also in text classification \citep{lewis1994comparison}, \citep{moulinier1996text}. It is a feature space pruning technique that measures the number of bits of information obtained(entropy) for category prediction by knowing the presence or absence of a term in a document. For terms where the information gain was below some predefined threshold are not considered in the document vector representation.  The information gain of a term $t$ is defined as
\begin{equation}
G(t) = -\sum_{i=1}^{|C|} P(c_{i})\log P(c_{i}) + P(t)\sum_{i=1}^{|C|} P(c_{i}|t)\log P(c_{i}|t) + P(~t)\sum_{i=1}^{|C|} P(c_{i}|~t)\log P(c_{i}|~t)
\end{equation}

\item{\textbf{Mutual Information} : }Similar to the Information Gain scheme, Mutual Information estimates the information shared between a term and a category and prunes terms that are below a specific threshold. The mutual information between a term $t$ and a category $c$ is estimated in the following fashion, 
\begin{equation}
I(t,c) = \log \frac{P(t \wedge c)}{P(t) \times P(c)}
\end{equation}
To measure the goodness of a term in global feature selection, the category specific scores of a term are combined using, 
\begin{equation}
I_{avg}(t) = \sum_{i=1}^{|C|} P(c_{i})I(t,c_{i})
\end{equation}

\item{\textbf{$\chi^{2}$ Statistic} : }The $\chi^{2}$ statistic measures the lack of independence a term $t$ and a category $c$ and can be compared to the $\chi^{2}$ distribution with one degree of freedom. The term-goodness factor is calculated for each term-category pair and is averaged as above. The major difference between Mutual Information and $\chi^{2}$ statistic is that the later is a normalized value and the goodness factors across terms are comparable for the same category.

\item{\textbf{Latent Semantic Indexing (LSI)} : } LSI first introduced by \cite{deerwester1990indexing}, is a popular linear algebraic dimensionality reduction technique that uses the term co-occurence statistics to capture the latent semantic structure of the documents and represent them using low-dimensional vectors. It is an efficient technique to deal with synonymy and polysemy. LSI aims to find the best subspace approximation to the original document bag-of-word vector space using Singular Value Decomposition. Given a term-document matrix $X = \left[ x_{1}, x_{2}, \ldots, x_{|D|} \right] \in \mathbb{R}^{|V|}$, its k-rank approximation as found using SVD, can be expressed as, 
\begin{equation}
X = T S D^{T}
\end{equation}
where, $T \in \mathbb{R}^{|V| \times k}$ and $D \in \mathbb{R}^{|D| \times k}$ are orthonormal matrices called the left and right singular vectors respectively. The matrix $S \in \mathbb{R}^{k \times k}$ is a diagonal matrix of singular values arranged in descending order. The $k$-dimensional rows of the matrix $D$ contain the dimensionality reduced representations of the $|D|$ documents in the collection. The representations obtained using LSI alleviate the issue of data sparsity and high-dimensionality in bag-of-words representations and also helps unfold the latent semantic structure of the documents.
\end{enumerate}

\section{Learning Algorithms}
\label{sec:lalgos}
Multi-label text classification has seen growing number of statistical learning methods being applied to it. Over the years, various larning algorithms like, Regression models (\citep{cooper1994full}, \citep{fuhr1991air}), Conditional Random Field (\citep{ghamrawi2005collective}), Nearest Neighbour techniques (\citep{yang1994expert}, \citep{zhang2005k}, \citep{zhang2007ml}), Bayesian classifier and topic modelling (\citep{lewis1994comparison}, \citep{mccallum1999multi}, \citep{nigam2000text}, \citep{rubin2012statistical}, \citep{nigam1999using}, \citep{ueda2002parametric}), SVM (\citep{joachims1998text}, \citep{elisseeff2001kernel}), Neural Networks (\citep{wiener1995neural}, \citep{ng1997feature}), Decision Trees (\citep{tong1994machine}), Online learning algorithms (\citep{lewis1996training}, \citep{crammer2002new}), Non-negative Matrix Factorization (\citep{liu2006semi}) etc. have been used or developed for Multi-label document categorization.

Earlier learning algorithms reduced the problem of multi-label classification into multiple binary classification problems and independently learned binary classifiers for each category. While these algorithms performed well, their drawback of considering correlation among categories led to the development of algorithms that learn a single classifier and jointly classify each document. 

Multi-label classification problems can be also be classified into classification-based and ranking-based approaches, where the former assigns each test instance a $|L|$-sized label vector of ones and zeros indicating the presence and absence of labels. In the case of a ranking-based approach, the ranking system outputs the list of labels arranged in the increasing order of a ranking score which is then thresholded at an optimum and the top labels are considered appropriate label assigments for test instances.

Below we describe some of the famous learning algorithms for multi-label text classification, 

\subsection{With Multiple Binary Classifiers}
The most common approach of multi-label text classification, treats each label independently and learns multiple binary classifiers, one for each category and then assigns to a test document all the categories for which the corresponding classfier says \emph{`yes'}. In this section we describe some of the algorithms in the context of multi-label text classification.
\begin{enumerate}
\item{\textbf{Logistic Regression (LR)} : }Introduced by \citep{hosmer1989applied}, LR is a probabilistic binary classification regression model, that, for binary text classification learns a category weight vector and estimates the probability of a document belonging to the category using dot-product and the logistic link function. LR can be extended for multi-label document classification by learning multiple category vectors, specifically, one for each category. At test time, one would need to query all category vectors for each document to make the category assignments. In our work, we use logistic regression for multi-label text classification. The details for the model are given in \todo{Future link to description}.

\item{\textbf{Support Vector Machines (SVM)} : } Support Vector Machines (\citep{cortes1995support}, \citep{vapnik2000nature}) based on the \emph{Structural Risk Minimization} principle, are universal learners. In their basic form, SVMs learn linear threshold functions to find linear hyperplanes in the input data space to separate data of the two differnt classes. In the case, where data is not linearly separable, SVMs can be plugged-in with appropriate kernel functions to learn polyniomial classifiers, radial basic fucntions etc. For multi-label text classification, training data is treated separately for each category and maximum margin separating hyperplanes are found for each category independently \citep{joachims1998text}.

\cite{elisseeff2001kernel} study a ranking based variant of SVM, where the positive/negative distance from the separating hyperplane of a specific category is the score assigned to the particular instance for that category. Their formulation then aims to maximize the margin between the score of a category that belongs to the document and a category that does not belong to do the document. This is also called the Rank-SVM.

\item{\textbf{Neural Networks (NNet)} : }Classification-baed, Neural Network approaches to multi-label text classification were mainly studied by \cite{wiener1995neural}, developed at Xerox PARC and called NNet.PARC and \cite{ng1997feature}, called CLASSI. Both neural networks are examples of multiple-classifier based approaches where a separate neural network was trained for each category to make binary classifications. While CLASSI used a linear perceptron approach to classify text into categories, NNet.PARC built a three-layered nonlinear neural network that extends logistic regression by modelling higher order term interactions and hence finding non-linear decision boundaries. 

\item{\textbf{Naive Bayes (NB)} : }Naive-bayes as studied in \cite{lewis1992representation} and \cite{lewis1994comparison}, is one of the most effective and simple statistical model for text classification. For multi-label classification, classifiers are learnt so as to estimate $P(C_{j}=1|D)$, i.e., the probability that the document, $D$ belongs to the category $C_{j}$, for each category. This probability is estimated by estimating the probability $P(W_{i}=1|C_{j}=1)$, i.e. probability that a particular word appears in the document when it belongs to a particular category. Though this approach makes the assumption of word independence, experiments show that this fast-learning algorithm can yield excellent results. 
\end{enumerate}
Although, approaches to multi-label classification discussed above give competitive accuracies in the task, they suffer from inefficiencies due to the following reasons,
\begin{itemize}

\item make assumptions of category independence and learn 1-vs-All binary classifiers. It is realized that such assumption would not hold true in most real-life situations. Fine-grained categorization of texts usually involve strongly correlated category classes and information about the presence of one gives information about the presence/absence of many others. For eg. in the sentence, 
\begin{quote} 
\centering 
\emph{Chicago Board of trade grain traders and analysts voiced a lot of interest in how farmers planned to handle their upcoming spring plantings prompting sales of new crop months of corn and oats and purchases in new crop soybeans in the futures markets}
\end{quote}
information from words about the presence of categories like \emph{oats}, \emph{corn} etc. can also aid the prediction of the \emph{agriculture} category which can be boosted using joint classification.

\item Apart from inefficiencies induced by ignoring category correlations, learning independent classifiers poses other drawbacks, such as, in case of millions of labels, learning millions of high-dimensional classifiers is a computationally expensive. Secondly, the cost of prediction for each test instance would be high as all the classifiers need to be evaluated to make a single prediction.
\end{itemize}

\subsection{With Single Joint Classifier}
To overcome the difficulties in learning multiple binary classifiers, researchers have since developed learning algorithms that jointly classify each document into categories it belongs to. Outputs of such algorithms are $|L|$-size label vector with each element being either of $\{0, 1\}$, indicating the presence/absence of a category. 
